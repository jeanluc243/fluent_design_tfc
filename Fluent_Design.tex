\documentclass[a4paper,12pt]{report}

\usepackage[french]{babel}
\usepackage[T1]{fontenc}
\usepackage[utf8]{inputenc}
\usepackage{lmodern}
\usepackage{microtype}

\usepackage{graphicx}
\graphicspath{ {figures/} }
\usepackage{array}

\usepackage{hyperref}

%%%%%%%%%%%%%%%%%%%%%%%%%%%%%%%%%%%%%%%%%%%%%%%%%%%%%%%%
\begin{document}
    %%%%%%%%%%%%%%%%%%%%%%%%%%%%%%%%%%%%%
    %%
    %% Page de Garde
    %%
    %%%%%%%%%%%%%%%%%%%%%%%%%%%%%%%%%%%%%

    \begin{titlepage}
        % En tete 
        \begin{center}
            {\Large ÉCOLE SUPÉRIEURE D’INFORMATIQUE SALAMA}\\
            {\large République Démocratique Du Congo}
            
            {\large Province de Haut-Katanga}
            
            {\large Lubumbashi}
            
            {\large www.esisalama.org}
            \vspace{10pt}
            
            \rule[30pt]{300pt}{1pt}
        \end{center}
        % Logo
        \begin{center}
            \includegraphics[width=100]{images/logo.png}
        \end{center}
        
        \vspace{15px}

        \begin{center}
            
            \rule[10pt]{\textwidth}{1pt}
            {\LARGE IMPLÉMENTATION DU FLUENT DESIGN AVEC FLUTTER }
            \bigskip
            % \hspace{10pt}
            \rule[10pt]{\textwidth}{1pt}
        \end{center}

        \begin{flushright}
            \small{
                Travail présenté et défendu en vue de l’obtention\\ du grade
            d’ingénieur technicien en Génie Logiciel .  
            }
            
        \end{flushright}

        \hspace{10pt}

        \begin{flushright}
            \textbf{\em{ {\small Par : KABULU MBOLELA Jean Luc}}}
            \\
            \textbf{\em{ \small{Option : Génie Logiciel }}}
        \end{flushright}
        \hspace{5pt}
        \begin{flushright}
            \textbf{\em{ {\small Directeur : Père KAMIBA Isaac }}}
            \\
            \textbf{\em{ \small{Co-directeur : MUKANDA KENGWE Henrique}}}
        \end{flushright}

        \vspace{30pt}
        \begin{center}
           {\large Mars 2020}
        \end{center}

    \end{titlepage}

    \newpage
    
    \begin{center}
        \Huge{IMPLÉMENTATION DU FLUENT DESIGN AVEC FLUTTER}
        \\
        \vspace{15pt}
        \large{KABULU MBOLELA Jean Luc}
        \\
        \vspace{10pt}
        \large{Génie Logiciel}
        \\
        \large{ESIS 2019-2020}
    \end{center}

    \newpage

    %%%% Épigraphe
    

    \begin{center}
        % {\huge \textbf{Épigraphe}}
        \chapter*{Épigraphe}
        \addcontentsline{toc}{chapter}{Épigraphe}
        \vspace{50pt}
         "Celui qui agit d'une main lâche devient pauvre, mais la main des diligents enrichit."
        \begin{flushright}
            Proverbes 10:4, Bible Darby.
        \end{flushright}
    \end{center}

    
    %%%%% Dédicace

    \chapter*{Dédicace}
    \addcontentsline{toc}{chapter}{Dédicace}
    
    \begin{em}
        À mes très chers parents MBOLELA WA KANYANA Patrick \-et CIBOLA Agnès.
        \\
        À mes frères MUKADI MBOLELA Serges, NSONA MBOLELA Cedrick, MUKENGESHAYI MBOLELA Idriss,
        NGOY MBOLELA Daniel et mes sœurs BILONDA MBOLELA Irene, NGALULA MBOLELA Rachel.
        \\
        À ma dulcinée FATIMA FURAHA Espérance.    
    \end{em}
        
    

    %%%%%% Remerciements
    
    \chapter*{Remerciements}
    \addcontentsline{toc}{chapter}{Remerciements}
    

    Arrivé au terme de notre cycle de formation d’Ingénieur Technicien à l’Ecole Supérieure d’Informatique Salama, 
    E.S.I.S en sigle, il est de notre devoir d’exprimer notre gratitude aux personnes qui nous ont aidé à 
    arriver là où nous en sommes aujourd’hui et à réaliser ce travail.
    \\
    C’est ainsi qu’en premier lieu, nous remercions \textbf{le seigneur Dieu} pour tout ses bienfaits.
    \\
    
    Nos remerciements s’adressent à tout le corps administratif et professoral de l’Ecole
    Supérieure d’Informatique Salama pour avoir concouru à notre formation.
    \\
    De manière particulière au Père KAMIBA Isaac et à Monsieur MUKANDA KENGWE Henrique,
    en leurs qualités respectives de Directeur et Co\-directeur de notre travail.
    \\
    À mes très chers parents MBOLELA Patrick et CIBOLA Agnès.
    \\
    À mon mentor le pasteur Roland DALO.
    \\
    Aux différents communautés dans lesquels je fut leader: Google Developper Student Clubs, Github Campus Expert,
    Microsoft Student Partner, AWS Educate Cloud Ambassador, Flutter Lubumbashi Community, Espacesis.
    \\
    À mes frères MUKADI MBOLELA Serges, NSONA MBOLELA Cedrick, MUKENGESHAYI MBOLELA Idriss,
    NGOY MBOLELA Daniel et mes sœurs BILONDA MBOLELA Irene, NGALULA MBOLELA Rachel.
    \\

    À tous mes compagnons de lutte avec qui nous avons passé notre parcours
    académique: BILEU KAPEPULA Shekina, KYUNGU LUPUNDU Dan, ILUNGA LUBABA Nathan, MANANG KAFUTSHI Elsa, 
    KASY NDIJI Laura, 
    AMPIRE BIGOMOKERO Eric et à toute la promotion 2019-2020 de G3 Génie Logiciel.
    \\
    À mes amis: NGALULA KALENDA Jenny, NGOIE MULOLO Carel et BILEU KAPEPULA Shekina.
    \\
    % À tous mes frères et sœurs, neveux et nièces, amis et connaissances dont les noms
    % ne sont pas cités, mais que nous portons chaleureusement dans notre cœur.
    % Trouvez ici, l’expression de ma profonde gratitude.

    \newpage

    %%%%%% Liste des figures
    \listoffigures
    \newpage

    %%%%%% Liste des tableaux
    \listoftables
    \newpage

    %%%%% Table des matières
    \tableofcontents
    \newpage

    %%%%% Avant-Propos
    \chapter*{Avant-Propos}
    \addcontentsline{toc}{chapter}{Avant-propos}
    Ce travail rentre dans le cadre de l’obtention du diplôme d'ingénieur en Génie Logiciel. 
    Il montrera comment nous sommes parvenus aa trouver une solution à l'utilisation du Fluent Design lors du développement des applications mobiles. L’idée de ce travail de fin de cycle est venue du constat que les développeurs passent beaucoup à concevoir des interfaces graphiques des applications, mais qui, souvent ne respectant les principes d'ergonomie.
    
    
    En Effet, depuis 2018, l'utilisation de Flutter semble augmenter d'une façon étonnante, les développeurs migre de plus en plus vers l'utilisation de Flutter à cause de sa facilite en conception mobile moderne. 
    Flutter est devenu la nouvelle solution de developpement des applications mobiles soit Android ou encore iOS. Entre Temps Flutter, apporte une expérience mobile plus simples et plus moderne dans l'utilisation des applications mobiles.
    Ce travail se veut être une contribution devant permettre de mettre en relief les différents obstacles, mais aussi les avantages d'utilisation de Fluent Design avec Flutter. Ainsi, une bibliothèque est proposée pour lever ces obstacles, en particulier sur Android et sur iOS.
    
    
    Des difficultés n’ont pas manqué. Elles concernent particulièrement la disponibilité de données fiables et actuelles. Elles concernent également la disponibilité des designers qui prennent les décisions dans la conception des applications mobiles pour la réalisation d’interview. Cette dernière situation nous a obligé à nous contenter des entretiens informels que nous avons pu avoir avec quelques spécialistes.

    \vspace{20px}
    \begin{flushright}
        Ce document a été traité avec \LaTeX
    \end{flushright}
    
    %%%%%% Introductio
    \chapter*{Introduction}
    \addcontentsline{toc}{chapter}{Introduction}

        \section*{Problématique}
        \addcontentsline{toc}{section}{Problématique}

            \subsection*{Situation de la problématique}
            \addcontentsline{toc}{subsection}{Situation de la problématique}

                De nos jours, les utilisateurs de téléphones portables attendent de leurs applications un beau design, 
                des animations fluides et de grandes performances. 
                De plus en plus avec le développement en intégration continue qui oblige aux entreprises ou des 
                startups des développement des applications de publier d’une manière constante les mises a jours .

                
                Entre temps pour permettre aux utilisateurs des applications mobiles d’avoir plus des facilité et plus 
                d'intuition, de grandes Firmes telle que Microsoft, Google ou Samsung font des études approfondies sur 
                le comportement humains pour savoir comment ils réagissent face une applications mobiles, 
                Comment ils comprennent différentes icônes, comment ils déplacent leurs doigts sur l'écran , 
                comment ils réagissent devant certaines couleurs.
        
                Depuis Décembre 2019, Microsoft a lancé "Fluent Design System" \footnote{https://www.microsoft.com/design/fluent/} ,un ensemble des principes fondamentaux 
                de Design pour aider les concepteurs et développeurs à construire et concevoir des produits 
                pour leur clients  visant à créer de la simplicité et de la cohérence grâce à un système de conception 
                partagé et ouvert sur toutes les plateformes.
        
                Fluent Design est un système de design collective et ouverte qui garantit que les personnes, 
                les équipes et leurs produits disposent des composants et des processus fondamentaux pour créer 
                des expériences utilisateurs cohérentes sur toutes les plateformes.
        
            \subsection*{Problème de la recherche}
            \addcontentsline{toc}{subsection}{Problème de la recherche}
                Les problèmes que nous avons rencontré est que les développeurs manque du code qui contient l’ensemble des
                 principes et des outils ( tels que la typographie, les couleurs, Les contrôles, Les animations )  
                 qui permet d'implémenter le Fluent Design dans leur applications mobiles.

            \subsection*{Question de la recherche}
            \addcontentsline{toc}{subsection}{Question de la recherche}
                En lisant les lignes qui précèdent, la question suivante peut être soulevée :
                
                \begin{itemize}
                    \item comment concevoir une applications avec moins de code pour avoir une application 
                    qui respecte l'aspect visuel de Fluent Design ?
                \end{itemize}
        
        \section*{Hypothèses}
        \addcontentsline{toc}{section}{Hypothèses}

            Face à ce problème auquel se confronte les développeurs, principalement les développeurs des applications mobiles 
            qui aimerait utiliser le fluent Design, nous avons  mis au point une bibliothèque native qui fournit 
            l'expérience de l'interface utilisateur de Microsoft pour la plate-forme Android.
            pour y parvenir notre bibliothèque apporte du style pour les boutons, les formulaires, 
            la navigation… Il permet ainsi de concevoir une application mobiles rapidement et avec peu de 
            lignes de code ajoutées.
        
        \section*{Choix et Intérêt du sujet}
        \addcontentsline{toc}{section}{Choix et Intérêt du sujet}
            
            \subsection*{Choix du sujet}
            \addcontentsline{toc}{subsection}{Choix du sujet}
                Comme le stipule le vad mecum de notre institut , 
                a la fin  du premier cycle à l'ecole supérieure d'informatique Salama, 
                il est demander de resoudre un probleme autour comme travail de fin de cycle.
                Notre choix a porté sur « L'implémentation du Fluent Design avec Flutter » 
                dans le but de faciliter le développement des applications mobiles.
            
            \subsection*{Intérêt du sujet}
            \addcontentsline{toc}{subsection}{Intérêt du sujet}

                \subsubsection*{Intérêt Personnel}
                \addcontentsline{toc}{subsubsection}{Intérêt Personnel}
                    Étant Largement passionné par le développement mobile et très émerveillé par la technologie Flutter \footnote{Flutter est le framework d'interface utilisateur de Google qui permet de créer de belles applications nativement compilées pour le mobile, le web et le pc à partir d'une seule base de code.} \footnote{http://flutter.dev/}, 
                    qui est devenu est une technologie phare et populaire, 
                    nous avons décidé de contribuer avec une bibliothèque qui aidera la communauté des développeurs mais aussi 
                    dans le but d'approfondir ma connaissance sur Flutter.
                
                \subsubsection*{Intérêt scientifique}
                \addcontentsline{toc}{subsubsection}{Intérêt scientifique}
                    Nous avons opté pour ce sujet afin de répondre aux exigences académiques fixées à tout étudiant finaliste, 
                    celui de réaliser un travail de fin de cycle. Mais l’obtention d’un diplôme n’est pas le seul l'intérêt de ce travail, 
                    car nous le voulons aussi pour contribuer aux développement et l'évolution du Framework Flutter.  
                    Nous voudrions que ce travail 
                    soit un ajout nécessaire qui permettra de développer plus rapidement les applications mobiles avec Flutter.
            
            \section*{Méthode}
            \addcontentsline{toc}{section}{Méthode}
                Une bonne méthodologie permet de structurer les démarches à suivre de façon à ce que l’etudiant puisse évoluer 
                dans le temps, en suivant les jalons importants aussi bien pour les travaux de recherches que pour la 
                rédaction du travail.

                Pour notre travail, nous avons opté pour la méthode nommée UP \footnote{UP : Unified Process},
                Le processus unifié est un processus de développement logiciel itératif, centré sur l'architecture, 
                piloté par des cas d'utilisation et orienté vers la diminution des risques.
            
            \section*{Techniques}
            \addcontentsline{toc}{section}{Techniques}
                Pour notre notre travail nous avons utilisé la technique suivante :
                \begin{itemize}
                    \item Documentaire qui consiste essentiellement à consulter des livres, 
                          des notes de cours, des documentations sur internet, des travaux de fin de cycle, 
                          des thèses soutenues et des articles voir des vidéos disponibles sur les différentes plateformes 
                          d’enseignements en ligne
                \end{itemize}

            \section*{Etat de l’art}
            \addcontentsline{toc}{section}{Etat de l’art}

                Après le lancement du Fluent Design \footnote{5 Déc. 2019: https://www.theverge.com/2019/12/5/20996748/microsoft-fluent-design-mobile-office-apps-new-updates-features}, les développeurs de Microsoft ont conçu 
                une bibliothèque qui permettrait d’utiliser deja ce norme, 
                et cette bibliothèque s’intitule : Office UI Fabric for Android \footnote{https://github.com/OfficeDev/ui-fabric-android}: 
                une bibliothèque croissante de contrôles écrits en Kotlin. 
                Ces contrôles mettent en œuvre le langage de conception Fluent et apportent une 
                cohérence entre les expériences des applications Office uniquement sur Android.
                
                
                Notre objectif n’est pas de reproduire la même bibliothèque, 
                mais plutôt de mettre en place une bibliothèque fonctionnant sur les mêmes bases, 
                en apportant notre nouveauté qui consiste à implémenter les animations, les contrôles 
                et la responsivité mais sur différentes plateformes : Android et iOS.
            
            \section*{Délimitation du travail}
            \addcontentsline{toc}{section}{Délimitation du travail}
                Notre travail sur la conception d’une bibliotheques Flutter absolument 
                ecrit en langage de programmation Dart, qui contient ensemble des 
                composants nécessaires à la réalisations d’une application Android ou iOS 
                qui respecte les normes du Fluent Design.

                Notre bibliothèque contient des informations sur les couleurs et la 
                typographie, ainsi que sur les contrôles personnalisés et les personnalisations 
                des contrôles selon les plateformes, le tout  en respectant les principes 
                officiels du  Fluent Design.
            
                Nous nous limiterons à la conception de la bibliothèque et à la conception 
                d’une application mobiles de démonstration de la bibliothèque mais aussi 
                du Fluent Design.
            
                
                
                

\end{document}